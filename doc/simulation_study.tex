% Project:     IMS project
% @file        doc/simulation_study.tex
% @date        29.11.2023
% 
% @brief Simulation study file
%
% @author Adam Ližičiar <xlizic00@stud.fit.vutbr.cz>

\documentclass[a4paper, 11pt]{article}

\usepackage[slovak]{babel}
\usepackage[utf8]{inputenc}
\usepackage[left=2cm, top=3cm, text={17cm, 24cm}]{geometry}
\usepackage{times}
\usepackage{verbatim}
\usepackage{enumitem}
\usepackage{listings}
\usepackage{color}
\usepackage{graphicx}
\usepackage{tabularx}
\usepackage[unicode]{hyperref}
\hypersetup{
    colorlinks=true,
    linkcolor=black,
    urlcolor=blue, 
    citecolor=blue
}
\definecolor{mygreen}{rgb}{0,0.6,0}
\definecolor{mygray}{rgb}{0.5,0.5,0.5}
\definecolor{mymauve}{rgb}{0.58,0,0.82}
\lstset{
  backgroundcolor=\color{white},
  basicstyle=\footnotesize,
  breaklines=true,
  captionpos=b,
  commentstyle=\color{mygreen},
  escapeinside={\%*}{*)},
  keywordstyle=\color{blue},
  stringstyle=\color{mymauve},
  tabsize=4
}


\begin{document}

	%%%%% Titulná stránka %%%%%
	\begin{titlepage}
		\begin{center}
			\includegraphics[width=0.77\linewidth]{res/logo_FIT.pdf} \\

			\vspace{\stretch{0.382}}

			\scalebox{2}{\Huge{Simulačná štúdia}} \\
			\LARGE{Transport tovaru tranzitnou spoločnosťou DaliTrans} \\
			\vspace{\stretch{0.618}}
		\end{center}

		\begin{minipage}[b]{0.4 \textwidth}
			\raggedright
			{\Large \today}
		\end{minipage}
		\hfill
		\begin{minipage}[b]{0.6 \textwidth}
			\raggedleft
			\Large
			Adam Ližičiar (xlizic00)\\
		\end{minipage}		
	\end{titlepage}

	%%%%% Obsah %%%%%
	\pagenumbering{roman}
	\setcounter{page}{1}
	\tableofcontents
	\clearpage

	\pagenumbering{arabic}
	\setcounter{page}{1}
	
	%%%%% Úvod %%%%%
	\section{Úvod}
	Práca sa~zaoberá rozvozom tovaru tranzitnej spoločnosti DaliTrans.
    Vďaka tomuto modelu a~simulačnému experimentu je~možné pozorovať
    efektívnosť aktuálneho systému a~nájsť spôsoby na~zefektívnenie
    tohto systému.\newline
    V~reálnom systéme je~náročné zisťovať ekonomické rozdiely, pretože
    systém obsahuje veľké množstvo entít a~faktorov, ktoré s~nimi súvisia.

    %%%%% Fakty %%%%%
	\section{Fakty}
	Informácie o~trantiznej spoločnosti boli získané z~jej oficiálnej
    webstránky, štatistických údajov spoločnosti a~následne validované
    majiteľom firmy Daliborom Janegom. 

    \subsection{Textová schéma modelu}
    Spoločnosť je~zameraná na~vnútroštátnu i~zahraničnú prepravu,
    preto model opisuje obdobie
    jedného mesiaca. K~aktuálnemu dátumu je~v spoločnosti
    267~tranzitných vozidiel (väčšina Renault Trucks
    520 T-High). Tieto vozidlá majú spotrebu pri~naloženom prívese
    34~litrov na~100 kilometrov a~pri~prázdnom prívese 22~litrov
    na~100 kilometrov.\newline
    Jazdy sú~už~vopred tak naplánované, že~po~dokončení jazdy je~šofér
    vyslaný na~ďaľšiu jazdu, ku~ktorej sa~dostane
    za~64~minút ±~23~minút.
    Dovolenky alebo ochorenia zamestnancov nie~je~potrebné riešiť
    z~dôvodu dostatočného počtu voľných zamestnancov a~taktiež
    podrobnému naplánovaniu jázd, kedy sa každý vodič vráti po~približne
    5~dňoch v~práci na~depo (kamión prevezme ďalší zamestnanec). Počet
    voľných zamestnancov nie~je~potrebné riešiť z~dôvodu, že~ich 
    je~dostatočný počet a~v~ojedinelom prípade, kedy by nikto zo~zamestnancov
    nebol dostupný, sú~volaní externí šoféri. Každý mesiac
    je~vykonaných 5083 jázd  ± 256 jázd. Pravdepodobnosť poruchy
    sú 2,3~\%. Jej následná oprava trvá 2 hodiny.\newline
    Prechod k zákazníkovi od miesta nakladania trvá priemerne 
    84~minút ±~23~minút. Následne prebehne naloženie (46~minút ±~21~minút).
    Po naložení sa vodič vyberie na jazdu, ktorá trvá 11 hodín ± 8 hodín
    (v~čase sú~započítané aj povinné pauzy podľa štandartov Európskej
    únie).
    Vykladanie zaberie 28 minút ± 14 minút. Pri~prevoze je~vždy v~prívese
    tovar iba od~jedného zákazníka.\newline

    %%%%% Koncepcia a spôsob riešenia %%%%%
	\section{Koncepcia a spôsob riešenia}
	todo 

    %%%%% Testovanie a experimenty %%%%%
	\section{Testovanie a experimenty}
	todo 

    %%%%% Záver %%%%%
	\section{Záver}
	todo 

	%%%%% Referencie %%%%%
	\section{Referencie}
	todo

	%%%%% Prílohy %%%%%
	\clearpage
	\section{Prílohy}

\end{document}
